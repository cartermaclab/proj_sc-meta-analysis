\clearpage
\makeatletter
\efloat@restorefloats
\makeatother


\begin{appendix}
\hypertarget{p-curve-disclosure-form}{%
\section{\texorpdfstring{\emph{P}-curve disclosure
form}{P-curve disclosure form}}\label{p-curve-disclosure-form}}

\begingroup\fontsize{9.5}{11.5}\selectfont

\begin{landscape}
\begin{ThreePartTable}
\begin{TableNotes}
\item \textit{Note.} 
\item KR = Knowledge of results; PD = Parkinson's disease; SC = Self-controlled
\end{TableNotes}
\begin{longtable}[l]{>{\raggedright\arraybackslash}p{2cm}>{\raggedright\arraybackslash}p{5.5cm}>{\raggedright\arraybackslash}p{3cm}>{\raggedright\arraybackslash}p{2cm}>{\raggedright\arraybackslash}p{6cm}>{\raggedright\arraybackslash}p{1.5cm}}
\caption{(\#tab:unnamed-chunk-6)Experiment information from papers included in the p-curve analysis.}\\
\toprule
Original paper & Quoted text from original paper indicated predicted benefit of self-control relative to yoked practice & Design & Key statistical result & Quoted text from original paper with statistical results & Result\\
\midrule
Andrieux, Danna \& Thon (2012) & "Thus, we hypothesized that a practice condition in which the learner could set the level of task difficulty would be more beneficial for learning than a condition in which this parameter was imposed." & Two cell & Difference in means & "A follow up analysis restricted to the first two blocks revealed a significant difference between groups, F(1, 36) = 4.85, p < .05, partial eta squared = .12. Self-controlled learners were significantly more accurate (M AE = 12.73 mm, SE = 1.57) than their yoked counterparts (M AE = 18.1 mm, SE = 1.87) after a 24-hr rest." & F(1, 36) = 4.85\\
\addlinespace
Andrieux, Boutin, \& Thon (2016) & "Two main reasons led us to expect that self-control of nominal task difficulty would enhance motor skill learning, and especially when introduced during early practice rather than during late practice." & Four cell (Full self-control, full yoked, self-control then yoked, yoked then self-control) & Difference in means & "Planned pairwise comparisons revealed that the self-control groups exhibited lower RMSE (SC + SC, SC + YO, and YO + SC groups) than their yoked group counterparts (YO + YO group), F(1, 44) = 14.02, p < .01." & F(1, 44) = 14.02\\
\addlinespace
Brydges, Carnahan, Safir \& Dubrowski (2009) & "We hypothesised that participants with self-guided access to instruction would learn more than participants whose access to instruction was externally controlled." & 2 (Control: self, yoked) X 2 (Goals: process, outcome) & Difference in means & "The self-process group performed better on the retention test than the control-process group (Fig. 1). This effect was significant for time taken, (F[1,23] = 4.33, P < 0.05)." & F(1,23) = 4.33\\
\addlinespace
Chiviacowsky (2014) & "We hypothesized that participants of the self-controlled group would show superior motor learning than yoked participants" & Two cell & Difference in means & "The Self group outperformed the Yoked group. The group main effect was significant, t(26) = 2.08, p = .04, d = .78." & t(26) = 2.08\\
\addlinespace
Chiviacowsky, Wulf, de Medeiros, Kaefer \& Tani (2008) & "Therefore, the purpose of the present study was to examine whether the learning benefits of self-controlled KR would generalize to children." & Two cell & Difference in means & "The self-control group had higher accuracy scores than the yoked group. This difference was significant, F(1, 24) = 4.40, p < .05." & F(1, 24) = 4.40\\
\addlinespace
Chiviacowsky, Wulf, Lewthwaite, \& Campos (2012) & "The potential benefits of self-controlled practice have yet to be examined in persons with PD...under the assumption that self-controlled practice would enhance the learning of the task..." & Two cell & Difference in means & "The self-control group was overall more effective than the yoked group. Time in balance was significantly longer for the self-control group, F(1, 26) = 4.25, p < .05." & F(1, 26) = 4.25\\
\addlinespace
Chiviacowsky Wulf, Machado \& Rydberg (2012) & "We predicted that self-controlled practice, in particular the ability to choose when to receive feedback, would result in more effective learning compared to a practice condition without this opportunity (yoked group)." & Two cell & Difference in means & "The day following practice, a retention test (without feedback) revealed lower AEs for the self-control group than the yoked group (see Figure 2, right). The group difference was significant, with F(1, 28)= 4.72, p < 0.05, eta squared =.14." & F(1, 28)= 4.72\\
\addlinespace
Hartman (2007) & "The primary aim of this study was to test whether there would exist a learning advantage for a self-controlled group, as opposed to a yoked control group, for learning a dynamic balance task." & Two cell & Difference in means & "To assess the relatively permanent or learning effects of practice with or without a self-controlled use of a balance pole, both groups performed a retention test on Day 3. The group effect was significant, F(1, 17)  = 8.29, p < .01, with the Self-control group outperforming the yoked group." & F(1, 17) = 8.29\\
\addlinespace
Kaefer, Chiviacowsky, Meira Jr. \& Tani (2014) & "...both self-controlled groups (introverts and extroverts) will achieve a level of activation that facilitates learning through the control of stimulation source (feedback) in comparison with the groups that do not have control over it." & 2 (Control: self, yoked) X 2 (Personality: introvert, extrovert) & Difference in means & "The groups’ main effects were detected on the factor "feedback type": Self-controlled groups performed better, F(1, 52) = 4.13, p < .05, compared with externally controlled groups" & F(1, 52) = 4.13\\
\addlinespace
Leiker, Bruzi, Miller, Nelson, Wegman \& Lohse (2016) & "We hypothesized that participants in the self-controlled group would show superior learning (i.e., better performance on retention and transfer tests) compared to the yoked group." & Two cell & Difference in means & "Controlling for pre-pest, there was a significant main effect of group, F(1,57) = 4.51, p = 0.04, partial eta squared = 0.07, such that participants in the self-controlled group performed better on the post-test than participants in the yoked group." & F(1,57) = 4.51\\
\addlinespace
Lemos, Wulf, Lewthwaite \& Chiviacowsky (2017) & "Independent of which factor the learner is given control over e or whether or not this factor is directly related to the task to be learned e the learning benefits appear to be very robust." & Two cell & Difference in means & "On the retention test, choice participants clearly outperformed the control group. The group main effect was significant, F(1, 22) = 88.16, p < 0.01." & F(1, 22) = 88.16\\
\addlinespace
Lessa \& Chiviacowsky (2015) & "...it was hypothesized that older adult participants of the self-group would demonstrate superior motor learning results, presenting faster task times on the speed cup-stacking task, when compared with participants in the yoked control group." & Two cell & Difference in means & "The analysis of the retention test revealed significant differences between groups, F(1,34) = 4.87, p < .05...with participants of the self-control group presenting faster task times compared to yoked participants." & F(1,34) = 4.87\\
\addlinespace
Lewthwaite, Chiviacowsky, Drews \& Wulf (2015; Exp. 1) & "In the present experiment, the choice learners were given was not related to task performance per se. Therefore, any learning benefits resulting from having, as opposed to not having, a choice would suggest that motivational factors are responsible for those effects." & Two cell & Difference in means & "On the retention test, during which white golf balls were used, the choice group showed significantly higher putting accuracy (36.8) than the yoked group (26.4), F(1, 22) = 7.31, p < .05" & F(1, 22) = 7.31\\
\addlinespace
Lewthwaite, Chiviacowsky, Drews \& Wulf (2015; Exp. 2) & "Given the potential theoretical importance of the finding in Experiment 1, we wanted to replicate it with another task and different type of choice." & Two cell & Difference in means & "On the retention test 1 day later, the choice group demonstrated significantly longer times in balance than the yoked group, F(1, 27) = 7.93, p < .01." & F(1, 27) = 7.93\\
\addlinespace
Lim, Ali, Kim, Choi \& Radlo (2015) & "It was expected that a self-controlled feedback schedule would be more effective for the learning and performance of serial skills for both acquisition and retention phases than a yoked schedule." & Two cell & Difference in means & "In the retention phase, there was a significant main effect for Group (F(1, 22) = 18.27, p < .05). The follow-up test indicated that the Self-controlled feedback group had higher performance (Cohen's d = 6.4) than the Yoked-feedback group during the retention test in both blocks." & F(1, 22) = 18.27\\
\addlinespace
Patterson, Carter \& Sanli (2011: Comparison 1) & "We expected that the structure of this self-controlled practice context would either add to or compromise the existing benefits attributed to a self-controlled practice context." & 2 (Control: self, yoked) X 3 (Structure: full, all, faded) & Difference in means & "Specifically, the Self-Self condition demonstrated less |CE| compared to their Yoked-Yoked counterparts. This main effect was significant, F(1, 18) = 8.06, p < .05." & F(1, 18) = 8.06\\
\addlinespace
Patterson, Carter \& Sanli (2011: Comparison 2) & "We expected that the structure of this self-controlled practice context would either add to or compromise the existing benefits attributed to a self-controlled practice context." & 2 (Control: self, yoked) X 3 (Structure: full, all, faded) & Difference in means & "The All-Self condition demonstrated less |CE| compared to the All-Yoked condition. This main effect was also statistically significant, F(1, 18) = 4.67, p < .05." & F(1, 18) = 4.67\\
\addlinespace
Patterson, Carter \& Sanli (2011: Comparison 3) & "We expected that the structure of this self-controlled practice context would either add to or compromise the existing benefits attributed to a self-controlled practice context." & 2 (Control: self, yoked) X 3 (Structure: full, all, faded) & Difference in means & "The Faded-Self condition demonstrated less |CE| compared to the Faded-Yoked condition, supported by a main effect for group, F(1, 18) = 5.78, p < .05." & F(1, 18) = 5.78\\
\addlinespace
Post, Fairbrother, Barros \& Kulpa (2014) & "It was hypothesized that learners in the SC group would demonstrate superior accuracy and form scores compared with the yoked group during the retention test." & Two cell & Difference in means & "The univariate ANOVA for retention revealed a significant group effect, F(1, 29) = 6.08, p = .020. The SC group had higher Accuracy scores the YK group" & F(1, 29) = 6.08\\
\addlinespace
Ste-Marie, Vertes, Law \& Rymal (2013) & "We hypothesized that the Learner Controlled group would show superior physical performance of the trampoline skills… compared to the Experimenter Controlled group." & Two cell & Difference in means & "A separate independent samples t-test showed that the Learner Controlled group had significantly higher performance scores compared to the Experimenter Controlled group at retention, t(58) = 3.21, p < .05, d = .753." & t(58) = 3.21\\
\addlinespace
Wulf \& Adams (2014) & "We asked whether giving performers an incidental choice would also result in more effective learning of exercise routines." & 2(Group: self-control, yoked) X 3 (Exercise: toe touch, head turn, ball pass) X 2 (Leg: left, right) mixed design with repeated measures on the final two factors & Difference in means & "On the retention test… the choice group showed fewer errors than the control group. The main effects of group, F(1,18) = 25.35, p < .001, was significant." & F(1,18) = 25.35\\
\addlinespace
Wulf \& Toole (1999) & "If the beneficial effects of self-control found in previous studies are more general in nature (i.e., some general mechanism responsible for these effects), learning advantage would also be expected for self-controlled use of physical assistance." & Two cell & Difference in means & "The main effect of Group, F(1,24) = 4.54, p < .05, was significant. Thus, allowing learners to select their own schedule of physical assistance during practice had a clearly beneficial effect on learning." & F(1,24) = 4.54\\
\addlinespace
Wulf, Clauss, Shea \& Whitacre (2001) & "Importantly, however, if self-control promotes the development of a more efficient movement technique, one should see greater movement efficiency, as indicated by delayed force onsets, in self-control as compared to yoked participants." & Two cell & Difference in means & "Whereas the self-control group demonstrated relative force onsets that, on average, occurred about half the distance between the center of the apparatus and the participant's maximum amplitude, the yoked group’s average force onset had already occurred after they had travelled less than 20\% of the distance to the maximum amplitude. This group difference was significant, F(1,24) = 4.43, p < .05." & F(1,24) = 4.43\\
\addlinespace
Wulf, Raupach \& Pfeiffer (2005) & "Thus, if the learning advantages of self-controlled practice generalize to observational practice, allowing learners to decide when they want to view a model presentation should result in enhanced retention performance, with regard to movement form and, perhaps, movement accuracy, compared to that of yoked learners." & Two cell & Difference in means & "Overall, the self-control group had higher form scores than the yoked group throughout retention. The main effect of group F(1,23) = 5.16, p < .05, was significant." & F(1,23) = 5.16\\
\addlinespace
Wulf, Iwatsuki, Machin, Kellogg, Copeland, \& Lewthwaite (2017) Exp 1. & "The purpose of the present experiments was threefold. First, we deemed it important to provide further evidence for the impact of incidental choices on motor skill learning. Given that self-controlled practice benefits for learning have frequently been interpreted from an information-processing perspective (e.g., Carter, Carlson, \& Ste-Marie, 2014; Carter \& Ste-Marie, 2016), with limited regard for rewarding-motivational explanations, further experimental evidence for learning enhancements through choices not directly related to the task seemed desirable (Experiments 1 and 2)." & Two cell & Difference in means & "On the retention test one day later, the choice group demonstrated higher scores than did the control group. The group effect was significant, F(1, 29) = 5.72, p < .05." & F(1, 29) = 5.72\\
\addlinespace
Wulf, Chiviacowsky \& Drews (2015) & "To summarize, we hypothesized that an external focus and autonomy support would have additive benefits for motor learning (i.e., retention and transfer performance), as evidenced by main effects for each factor." & 2 (Autonomy support: self, yoked) X 2 (Focus: external, internal) & Difference in means & "On the retention test, the main effect of Autonomy Support was significant, F(1, 64) = 6.98, p < .01." & F(1,64) = 6.98\\
\addlinespace
Ikudome, Kuo, Ogasa, Mori \& Nakamoto (2019; Exp. 2) & "Previous studies manipulating participants’ choice of variables relevant to the experimental task have indicated that such choices have a positive effect on motor learning due to deeper information processing by the participants. Based on these studies, it is possible that this positive effect would be observed regardless of participants’ levels of intrinsic motivation, because this type of choice would not induce a change in perceived locus of causality from internal to external." & 2(Choice: self, yoked) X 2 (Motivation: high, low) & Difference in means & "An ANCOVA indicated significant main effects of choice, F(1, 39) = 8.93, p = .005." & F(1,39) = 8.93\\
\bottomrule
\insertTableNotes
\end{longtable}
\end{ThreePartTable}
\end{landscape}
\endgroup{}
\end{appendix}
